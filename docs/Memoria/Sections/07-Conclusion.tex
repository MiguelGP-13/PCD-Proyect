% Conclusion
\section{Conclusiones}

En este trabajo se ha realizado un estudio comparativo entre un pipeline jerárquico y un modelo multiclase directo para la clasificación automática de lesiones cutáneas. En nuestro caso, el pipeline ha mostrado un mejor desempeño, especialmente en la separación entre lesiones benignas y malignas, lo que resulta clave para reducir errores clínicamente relevantes. Además, la segunda etapa multiclase basada en aprendizaje contrastivo supervisado y KNN sobre \textit{embeddings} ha alcanzado métricas similares a las de modelos convolucionales, aportando flexibilidad para la incorporación de nuevas clases malignas.

La incorporación de Grad-CAM ha reforzado la utilidad clínica del sistema, al proporcionar interpretabilidad y permitir que los especialistas validen las predicciones sobre la base de criterios visuales relevantes. De este modo, el modelo no solo mejora la precisión en la clasificación, sino que también se presenta como una herramienta práctica de apoyo al médico.

En conjunto, los resultados sugieren que la combinación de técnicas de transferencia de aprendizaje, estrategias de balanceo y enfoques contrastivos constituye una vía prometedora para el desarrollo de sistemas automáticos de apoyo al diagnóstico dermatológico. No obstante, será necesario validar estos hallazgos en otros datasets y explorar nuevas estrategias, como el \textit{few-shot learning}, para seguir mejorando la capacidad del sistema en escenarios clínicos reales.
