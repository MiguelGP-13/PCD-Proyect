% Introducción
\section{Introducción}

El cáncer de piel es uno de los tipos de cáncer más frecuentes a nivel mundial, siendo el melanoma su variante más agresiva y potencialmente mortal \cite{Apalla2017}. La detección temprana y la clasificación precisa de las lesiones cutáneas son fundamentales para mejorar el pronóstico y la eficacia del tratamiento. Sin embargo, el diagnóstico manual basado en imágenes dermatoscópicas presenta desafíos importantes debido a la similitud visual entre lesiones benignas y malignas, la variabilidad entre especialistas y la necesidad de experiencia clínica especializada. La creciente incidencia del melanoma y la limitada disponibilidad de especialistas en muchas regiones hacen que la implementación de sistemas automáticos de apoyo al diagnóstico pueda ser útil.

En los últimos años, las técnicas de aprendizaje profundo han demostrado ser prometedoras para automatizar la clasificación de lesiones cutáneas, alcanzando rendimientos comparables a los de dermatólogos expertos. Entre estas técnicas, el aprendizaje contrastivo ha emergido como una estrategia eficaz para generar representaciones robustas y semánticamente significativas, especialmente en contextos con datos etiquetados limitados y distribuciones desbalanceadas.

Este trabajo presenta un estudio comparativo sobre la clasificación automática de lesiones cutáneas utilizando el conjunto de datos HAM10000 \cite{Tschandl2018}. El uso de este dataset estandarizado permite evaluar de manera objetiva diferentes metodologías, fomentando la reproducibilidad y la comparación justa entre enfoques. En este contexto, el aprendizaje profundo no solo ofrece mejoras en precisión, sino también la posibilidad de desarrollar herramientas escalables e interpretables que contribuyan a una atención médica más equitativa y con potencial relevancia clínica como apoyo al diagnóstico dermatológico.

El código fuente utilizado en este trabajo, junto con la reproducción de resultados, está disponible públicamente\footnote{\url{https://github.com/MiguelGP-13/PCD-Proyect}}.

\vspace{1cm}
% Contribuciones
\subsection{Contribuciones}

Las principales contribuciones de este trabajo son las siguientes:

\begin{itemize}
  \item Se realiza un estudio comparativo entre un pipeline jerárquico y un modelo multiclase directo, evaluando su impacto en la clasificación de lesiones cutáneas.
  \item En la primera etapa del pipeline jerárquico, se aborda la clasificación binaria entre lesiones benignas y malignas, con especial énfasis en mejorar la detección de las lesiones malignas, que se encuentran infrarepresentadas en el dataset.
  \item Se aplican técnicas de reponderación y aumento de datos para mitigar el desequilibrio de clases y mejorar la sensibilidad hacia las categorías minoritarias.
  \item En la segunda etapa, se comparan enfoques basados en aprendizaje contrastivo frente a alternativas no contrastivas para la clasificación multiclase de las lesiones malignas, analizando su capacidad para generar representaciones discriminativas.
  \item Se realiza un análisis visual de los \textit{embeddings} para explorar relaciones entre lesiones malignas, formación de clústeres e identificación de casos ambiguos.
  \item Se incorpora Grad-CAM (Gradientweighted Class Activation Mapping) como herramienta de interpretación, permitiendo visualizar las regiones más relevantes en las predicciones y facilitar la validación clínica.
\end{itemize}
