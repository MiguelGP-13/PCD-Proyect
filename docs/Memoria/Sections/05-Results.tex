% Results
\section{Resultados}
\subsection{Comparación resultado final}
Aunque la diferencia en rendimiento no es muy grande, los resultados muestran que, contrariamente a lo esperado, donde un modelo convolucional multiclase debería aprovechar mejor toda la información disponible, el enfoque basado en pipeline ofrece un mejor rendimiento. No solo permite la incorporación de nuevas clases malignas de manera más flexible, sino que también consigue un resultado superior en comparación con el modelo convolucional directo.
\vspace{0.5em}
\begin{table}[!htpb]
\centering
\begin{tabularx}{\linewidth}{p{0.35\linewidth} X X X}
\toprule
\textbf{Modelo} & \textbf{Accuracy} & \textbf{F1-score} & \textbf{Recall} \\
\midrule
Pipeline & \textbf{0.74} & \textbf{0.76} & \textbf{0.74} \\
Modelo completo & 0.68 & 0.72 & 0.68 \\
\bottomrule
\end{tabularx}
\caption{Comparación de rendimiento en la clasificación final.}
\label{tab:comparacion_binaria}
\end{table}
\FloatBarrier

\begin{figure}[h]
    \centering
    \includegraphics[width=1\linewidth]{Figures/ConfusionFinal.png}
    \caption{Comparación de matrices de confusión.}
    \label{fig:gradcam_maligno}
\end{figure}
\FloatBarrier
Se observa en las matrices de confusión que el pipeline comete menos errores, especialmente en la separación entre lesiones malignas y benignas, que es la más relevante. El principal fallo del pipeline que tiende a predecir melanoma más de lo que debería. Esto se debe a que es la clase con más muestras dentro de las malignas.

\subsection{Pipeline}
\subsubsection{Clasificación binaria}

Los resultados muestran que los modelos basados en transferencia de aprendizaje superan claramente al entrenamiento desde cero. El mejor desempeño se obtuvo con VGG16. Pero, ResNet50 destaca categorizando bien el \textit{Recall en malignas} del 85\% frente al 75\% de VGG16 en las lesiones malignas. Esto implica que comete pocos falsos negativos, es decir, que hay pocas manchas malignas que clasifique como benignas, que es el objetivo más importante.

\vspace{0.5em}
\begin{table}[!htpb]
\centering
\begin{tabularx}{\linewidth}{p{0.35\linewidth} X X X}
\toprule
\textbf{Modelo} & \textbf{Accuracy} & \textbf{F1-score} & \textbf{Recall en malignas} \\
\midrule
Desde cero & 0.79 & 0.77 & 0.79 \\
VGG16 (transfer) & \textbf{0.87} & \textbf{0.87} & 0.75 \\
ResNet50 (transfer) & 0.82 & 0.83 & \textbf{0.85} \\
InceptionV3 (transfer) & 0.77 & 0.79 & 0.61 \\
\bottomrule
\end{tabularx}
\caption{Comparación de rendimiento en la clasificación binaria.}
\label{tab:comparacion_binaria}
\end{table}
\FloatBarrier

\subsubsection{Clasificación de clases malignas}

En la segunda etapa del pipeline se evaluaron tres variantes para la clasificación multiclase de las lesiones malignas: contrastivo por pares, contrastivo supervisado y no contrastivo.  

\begin{table}[!htpb]
\centering
\begin{tabularx}{\linewidth}{p{0.35\linewidth} X X X}
\toprule
\textbf{Enfoque} & \textbf{Accuracy} & \textbf{F1-score} & \textbf{Recall} \\
\midrule
Contrastivo supervisado & \textbf{0.82} & 0.80 & \textbf{0.82} \\
No contrastivo & 0.81 & 0.80 & 0.81 \\
\bottomrule
\end{tabularx}
\caption{Comparación de rendimiento en la clasificación multiclase de lesiones malignas.}
\label{tab:comparacion_multiclase}
\end{table}
\FloatBarrier
Ambos enfoques obtuvieron resultados muy similares. En cambio, el contrastivo por pares no convergió, por lo que se quedaría en el 30\% de accuracy.

\subsection{Visualización de embeddings}

\begin{figure}[h]
    \centering
    \includegraphics[width=\linewidth,height=5cm,keepaspectratio]{Figures/Embeddings.png}
    \caption{Comparación de \textit{embeddings}.}
    \label{fig:gradcam_maligno}
\end{figure}
\FloatBarrier

Para evaluar la calidad de las representaciones aprendidas, se realizó un análisis visual de los \textit{embeddings} generados en la etapa multiclase. Se puede ver que la clase \textit{melanoma}, la más dominante en el conjunto de datos, forma un clúster más definido y separado respecto a las demás categorías malignas. Sin embargo, las clases \textit{carcinoma basocelular} y \textit{queratosis actínica} presentan una mayor superposición y sobretodo dispersión, lo que dificulta su diferenciación y genera casos ambiguos.

Este comportamiento sugiere que, aunque el aprendizaje contrastivo supervisado contribuye a mejorar la discriminación entre clases, la capacidad de separación sigue siendo limitada para las categorías minoritarias. Probablemente esto se deba a que el modelo que genera los \textit{embeddings} no era lo suficientemente profundo, o que el \textit{batch} debería ser más grande, para ser capaz de capturar patrones más sutiles en las representaciones latentes. 

\vspace{1.5cm}
En términos clínicos, la identificación de clústeres bien definidos resulta útil para reforzar la confianza en las predicciones, mientras que la detección de regiones de solapamiento permite señalar casos que requieren una revisión más cuidadosa por parte del especialista. De este modo, el análisis visual de \textit{embeddings} no solo aporta información técnica sobre el rendimiento del modelo, sino que también contribuye a su validación clínica al destacar posibles fuentes de error o incertidumbre.

\subsection{Grad-CAM}
\begin{figure}[!htpb]
    \centering
    \begin{minipage}{0.49\linewidth}
        \centering
        \includegraphics[width=\linewidth]{Figures/GradCamBenigna.png}
        \caption{Visualización con Grad-CAM sobre una mancha benigna}
        \label{fig:gradcam_benigna}
    \end{minipage}\hfill
    \begin{minipage}{0.49\linewidth}
        \centering
        \includegraphics[width=\linewidth]{Figures/GradCamMaligna.png}
        \caption{Visualización con Grad-CAM sobre una mancha maligna.}
        \label{fig:gradcam_maligna}
    \end{minipage}
\end{figure}
\FloatBarrier

Las visualizaciones con Grad-CAM muestran que el modelo concentra su atención en regiones pigmentadas y estructuralmente complejas, coincidiendo con criterios clínicos utilizados por dermatólogos. Esto refuerza la interpretabilidad del sistema y su potencial utilidad como herramienta de apoyo diagnóstico al profesional, ayudándole a repasar ciertas zonas, para intentar asegurar que comete los mínimos errores posibles. Las zonas en rojo indican mayor relevancia para la predicción.
